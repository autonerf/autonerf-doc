\chapter{Eisen en wensen}

Om een goed werkend systeem te kunnen maken zijn er een aantal
minimale eisen waaraan een geselecteerd systeem aan moet voldoen.
Iedere verhoging heeft zo zijn consequenties verder in het traject.
De eisen en wensen staan op volgorde, wordt voor het bovenste item
een krachtigere oplossing gekozen dan moeten de onderstaande items
dat ook ondersteunen.

\begin{itemize}
	\item Resolutie camera, zie sectie \ref{sec:cam};
	\item RAM geheugen FPGA, zie sectie \ref{sec:RAM};
	\item Camera interface, zie sectie \ref{sec:caminter};
	\item Microcontroller interface, zie sectie \ref{sec:ucinter};
	\item Ontwikkelingsbord, zie sectie \ref{doc:ontwb};
\end{itemize}

\section{Resolutie camera}
\label{sec:cam}
De resolutie van de camera bepaald hoe goed een object gedetecteerd
kan worden. Ook bepaald deze hoe snel het systeem 1 frame kan verwerken.
Om niet te snelle hardware nodig te hebben om een acceptabele frame-rate
te halen moet de resolutie niet te hoog liggen. HD of hoger is bijvoorbeeld
helemaal niet nodig.
Als minimale resolutie wordt een $640*480$ geaccepteerd.

\section{RAM calculatie}
\label{sec:RAM}	
Met de onderstaande calculatie is de minimaal benodigde ruimte te berekenen.

\begin{equation}
C = A*D*R*X*Y
\end{equation}

Waarbij $Y$ staat voor het aantal pixels dat 1 afbeelding hoog is;
$X$ het aantal pixels dat 1 afbeelding breed is;
$R$ het aantal bytes per kleur van 1 pixel;
$D$ het aantal kleuren in 1 pixel;
$A$ het aantal plaatjes data het RAM moet kunnen opslaan;	
en $C$ staat voor de hoeveelheid ruimte in bytes.

Om een goede inschatting te kunnen maken is het aan te raden om uit
te gaan van 4 plaatjes in 8-bit RGB formaat. Op deze manier moet er
altijd voldoende ruimte zijn.

\section{Camera interface}
\label{sec:caminter}
Een camera interface heeft als enige eis dat hij zo eenvoudig mogelijk
dient te zijn. Dit i.v.m. de implementatie tijd. De focus van het project
ligt op de vision bewerking, niet op de acquisitie van beelden.
Denk bij interfaces aan I$^2$C, SPI en/of parallele interfaces. Houdt
hierbij rekening met het benodigde aantal I/O's. Deze moeten wel voldoende
aanwezig zijn op het ontwikkelingsbord.

\section{Microcontroller interface}
\label{sec:ucinter}
Als microcontroller interface is het verstandig om gebruik te maken van
een zogenaamde FSMC-bus. Althans, zo heet dit bij ST microcontrollers.
Texas instruments heeft ook een dergelijk concept, de EMIF-bus.
Bij NXP is dat dan weer EMC-bus.

De FSMC-bus maakt gebruik van een 16-bit data bus en daarbij ondersteunt
hij 26-bits adressen. Dat is een totaal van maximaal $2*2^26 = 134217728 bytes$.

Daarnaast maakt de bus gebruik van enkele bedieningslijnen, doorgaans 4 stuks.

Bij een $640x480 pixel$ camera is het nodig om rekening te houden met $\log _2 \left(\frac{640 \times 480 \times 3}{2} \right) \equiv 18,813 \approx 19-bits$. Dat betekend dat we voor de FSMC-bus minimaal $ 20 + 16 + 4 \equiv 40  lijnen$ nodig zijn.

\section{Ontwikkelingsbord}
\label{doc:ontwb}
Het juiste FPGA ontwikkelingsbord wordt gekozen op basis van de onderstaande criteria.

\begin{itemize}
	\item Bij voorkeur een ontwikkelingskit voor camera applicaties;
	\item FPGA bord moet het aantal I/O's kunnen aansturen;
	\item er dient voldoende RAM aanwezig te zijn, zoals gezegd in sectie \ref{sec:RAM};
	\item de FPGA dient genoeg LUTs en flip-flops te hebben;
	\item FPGA moet voldoende snelheid hebben om een frame te verwerken;
\end{itemize}
De laatste 2 punten zijn wat vaag. Daarvoor geld, des te hoger, des te beter!