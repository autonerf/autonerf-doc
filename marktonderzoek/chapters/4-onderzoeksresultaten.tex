\chapter{Onderzoeksresultaten}

Uit het onderzoek is gebleken dat vooral het aanbod van camera modules erg klein
is. De enige bruikbare module is de Raspberry Pi camera module die gebruik maakt
van een gepatenteerd protocol (Camera-Serial-Interface, of CSI) ontwikkeld door
MIPI. Om toegang te krijgen tot de specificaties van dit protocol is het nodig
lidmaadschap tot het MIPI aan te vragen wat 4000 dollar per jaar
kost\footnote{http://mipi.org/join-mipi/membership-model}. Dit is voor een
eenmalig project niet te betalen dus de optie om alleen de Raspberry Pi camera
module te gebruiken in combinatie met een FPGA valt af.

Wat wel een optie is, is het gebruiken van een Raspberry Pi als acquisitie in
combinatie met een bestaand of zelf ontwikkeld protocol om te communiceren met
de FPGA die in dit model zal dienen als filter. Een andere optie is om geheel
af te stappen van een digitale camera module en gebruik te maken van een camera
module die een analoog signaal (bijvoorbeeld PAL of NTSC) uitstuurt. Door
bijvoorbeeld een MAX9526\footnote{http://datasheets.maxim-ic.com/en/ds/MAX9526.pdf}
te gebruiken kan dit signaal gedigitaliseerd worden naar een signaal wat de FPGA
vervolgens kan uitlezen. Een bijkomend voordeel is dat een PAL camera goed 
verkrijgbaar is en er dus veel keus is tussen kwaliteit (lichtgevoeligheid, etc.).

Verder is het mogelijk om een USB stack te implementeren op een FPGA, echter is
dit geen wenselijke keuze omdat er dan veel werk gaat zitten in het correct
implementeren van deze stack in plaats van beeldverwerking (waar dit project om
zou moeten draaien). Daarnaast is het nog onbekend of de camera ook 
geïnitialiseerd dient te worden. Als dat het geval is, dan is het nagenoeg 
onmogelijk om dit te implementeren in de FPGA.

Nog een andere optie is het geheel weglaten van de FPGA en een Raspberry Pi
gebruiken met de bijbehorende camera module. Een van de dingen waar dan wel
rekening mee moet worden gehouden is het feit dat de Raspberry Pi niet snel is
wat problemen kan veroorzaken.

Verder zou het ook een optie kunnen zijn om het CSI protocol te \emph{reverse
engineeren} om er gebruik van te maken, ook dit is niet wenselijk, omdat hier
veel tijd in zou gaan zitten.
