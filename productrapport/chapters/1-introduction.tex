\chapter{Inleiding}
\label{chap:introduction}

% Waar gaat het verslag over, wat zijn de achtergronden van het project
% (in het kort)?
%
% Hoe is de logische opbouw van het verslag?

Dit verslag is geschreven in opdracht van de Hogeschool van Arnhem en Nijmegen
in het kader van de minor Embedded Vision Design (EVD) aan de opleiding Embedded
Systems Engineering (ESE). Het beschrijft hoe het project van de studenten is
ontwikkeld en waarom bepaalde ontwerp keuzes zijn gemaakt gedurende de
ontwikkeling van het product.

Het doel van het project was om een autonome \emph{Nerfgun} (zie figuur
\ref{fig:nerfgun}) te ontwikkelen. Dit product moet in staat zijn om een
\emph{marker} te herkennen en daar vervolgens een \emph{nerfdart} (figuur
\ref{fig:nerfdart}) op af te vuren. De officiele projectomschrijving was als volgt:

\begin{quotation}
\emph{Uiterlijk week 20 van het academisch jaar 2013-2014 zal een systeem
ontwikkeld zijn die autonoom een Nerf-gun kan afvuren op bewegende, gemarkeerde,
doelwitten. Het systeem moet gebruik maken van camera beelden om deze doelwitten
te vinden binnen een omgeving.}
\end{quotation}

Het systeem is gedoopt tot \bold{\autonerf}, een samenvoeging van \emph{autonoom}
en \emph{Nerf}. In dit verslag worden de resultaten van de verschillende
projectfasen besproken en nader toegelicht.

Het eerste deel dat zal worden beschreven is het functioneel ontwerp (hoofdstuk
\ref{chap:functional}). Hier zal functioneel beschreven worden wat het systeem
moet doen. Er zal verder niet worden ingegaan op hoe het systeem dit bereikt.
Het hoofdstuk Technische Specificatie (hoofdstuk \ref{chap:technical}) zal hier
verder op ingaan. Hoofdstuk \ref{chap:realisation} zal de realisatie van het
project beschrijven. Er zal besproken worden welke beslissingen er gemaakt zijn
en waarom deze gemaakt zijn. Uiteindelijk zullen de resultaten van het testen
(hoofdstuk \ref{chap:testing}) besproken worden. Daarnaast zal in hoofdstuk
\ref{chap:conclusion} een conclusie worden getrokken en zullen eventuele
aanbevelingen worden gedaan voor het verdere ontwikkeling van het product.
