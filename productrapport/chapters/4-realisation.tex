\chapter{Realisatie}

% detailontwerp met bijbehorende berekeningen (voedingsstromen, waarden van
% componenten); aansluitschema's van kabels en connectors; detailschema’s van
% de hardware en listings van de software worden in de bijlagen opgenomen

\section{Software}
\label{sec:softreal}
Om de juiste data uit een frame te halen is er de nodige filtering nodig. 
Het ontwerpen van deze filters doen we middels de stappen in het vision ontwerp.
Hieronder staat beschreven hoe het geïmplementeerd is, van frame acquisitie tot 
I/O aansturing.

\subsection{Acquisitie}
\label{sub:acqreal}
%To you Ramon

\subsection{Kleuren filter}
\label{sub:kleurfilt}
%To you Ramon

\subsection{Contrast stretch}
\label{sub:contstr}
Om een betrouwbaardere threshold te kunnen maken gebruiken we een contrast stretch 
operatie. Deze functie bepaald de laagste en hoogste waarde in het beeld en smeert 
deze uit over het complete grijswaarde gebied. Door een vermenigvuldigingsfactor te 
bepalen is het eenvoudig om dit te bereiken. De vermenigvuldigingsfactor zal 
uiteraard > 1 moeten zijn, maar kan ook een komma getal zijn. Float operaties zijn 
dus essentieel en daarbij is een FPU (Floating Point Unit) erg handig. Om de 
zware calculaties te verminderen wordt er gebruik gemaakt van een LUT (Look Up Table) 
die hetzelfde bereik heeft als de grijswaarde. Gezien er gebruik wordt gemaakt van 
een grijswaarde bereik van 256 moet de calculatie 256x worden uitgevoerd.
De verhouding wordt als volgt bepaald:
$R = \frac{D}{Max - Min}$
Waarbij R de vermenigvuldigingsfactor is;
D het grijswaarde bereik;
Max de maximale pixel waarde in het originele beeld;
Min de minimale pixel waarde in het originele beeld.

Daarna kan de LUT gevuld worden:
%<-- Pseudo code!
value = i * R;
if value > D then value = D
lut[i] = value;
%<-- End pseudo code!

\subsection{Threshold}
\label{sub:threshold}
