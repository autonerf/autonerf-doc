\chapter{Functioneel Ontwerp}
\label{chap:functional}

% systeembeschrijving op het hoogste niveau; wat moet het systeem doen,
% niet hoe; hoe ziet de gebruiker het systeem; systeem opdelen in
% functionele blokken en definiëren van de interfaces daartussen

In dit hoofdstuk is de functionele specificatie van het \autonerf project
gedefinieërd. De functionele eisen van het systeem worden beschreven en de
functionele architectuur van het systeem wordt besproken.

\section{Specificatie}

Het te ontwikkelen systeem moet voldoen aan de volgende, functionele, eisen:

\begin{enumerate}
    \item Het systeem moet plug-and-play zijn, er moet dus geen aparte computer
        (PC of Mac) nodig zijn om het systeem aan te sturen;
    \item Het systeem moet \emph{markers} herkennen en daar op richten en een
        Nerfdart op af vuren;
    \item Het systeem moet, om markers te herkennen, gebruik maken van real-time
        camera beelden;
    \item En het systeem moet in staat zijn om op bewegende doelwitten te vuren.
\end{enumerate}

\section{Architectuur}

De omschrijving van de architectuur van het \autonerf systeem begint bij het
globale input-process-output (IPO) schema (figuur \ref{fig:ipo-global}). Dit
schema toont de input van het systeem (de real-time camera beelden) en de output
van het systeem (het richten van de Nerfgun en het eventueel vuren van een dart).

\begin{figure}[H]
    \begin{center}
        \begin{tikzpicture}[font=\sffamily,
  every matrix/.style={ampersand replacement=\&,column sep=1cm,row sep=1cm},
  source/.style={inner sep=.3cm},
  process/.style={draw,thick,rounded corners,fill=yellow!10,inner sep=.3cm},
  sink/.style={source},
  datastore/.style={draw,very thick,shape=datastore,inner sep=.3cm},
  dots/.style={gray,scale=2},
  to/.style={->,>=stealth',shorten >=1pt,thick,font=\sffamily\footnotesize},
  every node/.style={align=center,font=\footnotesize}]

    \matrix {
        \node[source](in){Real-time camera beelden};
            \& \node[process](proc){Verwerken camera beelden};
            \& \node[sink](out){Commando's Nerfgun}; \\
    };

    \draw[to](in) -- (proc);
    \draw[to](proc) -- (out);
\end{tikzpicture}

    \end{center}
    \caption{Het globale input-process-output schema van het \autonerf systeem.}
    \label{fig:ipo-global}
\end{figure}

\pagebreak

Om de taak volledig uit te kunnen vuren bestaat het systeem uit een aantal
verschillende logische blokken (figuur \ref{fig:ipo-main}).

\begin{enumerate}
    \item De \texttt{reader} is verantwoordelijk is voor de acquisitie
        van frames vanuit de real-time feed van camera beelden;
    \item De \texttt{vision} is verantwoordelijk voor het detecteren van
        markers in een frame van de camera feed en het bepalen van de
        afwijking hiervan;
    \item De \texttt{controller} is verantwoordelijk voor het aansturen van
        de Nerfgun en het vuren van darts.
\end{enumerate}

\begin{figure}[H]
    \begin{center}
        \begin{tikzpicture}[font=\sffamily,
  every matrix/.style={ampersand replacement=\&,column sep=1cm,row sep=1cm},
  source/.style={inner sep=.3cm},
  process/.style={draw,thick,rounded corners,fill=yellow!10,inner sep=.3cm},
  sink/.style={source},
  datastore/.style={draw,very thick,shape=datastore,inner sep=.3cm},
  dots/.style={gray,scale=2},
  to/.style={->,>=stealth',shorten >=1pt,thick,font=\sffamily\footnotesize},
  every node/.style={align=center,font=\footnotesize}]

    \matrix {
        \node[process](read){Reader};
            \& \node[process](detect){Detector};
            \& \node[process](local){Localizer};
            \& \node[process](control){Controller}; \\
    };

    \draw[to](read) -- (detect);
    \draw[to](detect) -- (local);
    \draw[to](local) -- (control);
\end{tikzpicture}

    \end{center}
    \caption{De verschillende logische blokken in het \autonerf systeem.}
    \label{fig:ipo-main}
\end{figure}
