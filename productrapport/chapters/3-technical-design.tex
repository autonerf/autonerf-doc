\chapter{Technisch Ontwerp}

% architectuur van de gekozen oplossing (blokschema's, hiërarchische schema's,
% architectuurschema's, wat in hardware, wat in software); afwegen van
% alternatieven; globale oplossing van de belangrijkste of meest complexe
% deelfuncties en onderdelen, toestandsdiagrammen van (belangrijkste of meest
% complexe) functies, gekozen implementatie van datastructuren, eventueel
% pseudo-code; hoe worden de functies en onderdelen gerealiseerd

\section{Hardware}
\label{sec:hardware}

\subsection{Opzet}
\label{sub:opzet}
Het initiële ontwerp was een combinatie systeem van FPGA en een CPU. Door in 
de FPGA operatoren te implementeren moest er een hoge frame-rate gehaald kunnen 
worden. Mede door de communicatie complexiteit is er afgezien van een FPGA 
systeem en is er gekozen voor een snelle processor met Linux. Zo kan er gemakkelijk 
gebruik worden gemaakt van een USB camera.

\subsection{Ontwikkelingsbord}
\label{sub:devboard}
De markt voor Linux gebaseerde ontwikkelingsbordjes is redelijk uitgebreid. Voor 
dit project zijn er echter een aantal eisen.

\begin{itemize}
	\item 2 PWM uitgangen t.b.v. de pan en tilt servo's
	\item minimaal 11 GPIO uitgangen t.b.v. 10 darts + veiligheid
	\item zo hoog mogelijke kloksnelheid t.b.v. frame-processing
	\item genoeg geheugen voor minimaal 2 640*480 RGB frames
	\item een community t.b.v. resources
	\item het moet betaalbaar blijven
\end{itemize}

Gezien het erg belangrijk is om voldoende informatie te kunnen vinden over aansturing 
van onder andere hardware en PWM pinnen is er besloten om een BeagleBone Black te 
nemen. Met 6 PWM kanalen, 69 GPIO's, 512Mb DDR3 RAM en een 1GHz CPU is dit een prima 
apparaat om te gebruiken voor dit project.

\subsection{Cape}
\label{sub:cape}
Om de veiligheid van personen om het systeem te kunnen garanderen wordt er wederom 
gebruik gemaakt van een hardware-matig beveiligingssysteem. Bestaand uit een 
vereenvoudigde versie van de PC + Arduino combinatie.
De interface zal bestaan uit een 3-tal LEDs en een 2-tal knoppen. De 3 LEDs zullen 
de gebruiker laten weten of dat de applicatie aan staat of niet, of dat het apparaat 
kan schieten of niet en of dat er herladen moet worden.
Met de 2 knoppen kan het schieten en draaien hardware-matig geactiveerd en gedeactiveerd 
worden. Daarnaast zit er een knop op om aan te geven dat het systeem herladen is.
De interface dient als cape op de BeagleBone Black gemonteerd te worden. Verder zal de 
cape de nodige voedingsspanningen voorzien aan de kleppen, servo's en de BeagleBone Black 
zelf. Door de applicatie automatisch te laten starten op de BeagleBone Black is het doel 
een plug-and-play systeem te maken.