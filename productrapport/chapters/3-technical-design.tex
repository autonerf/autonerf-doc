\chapter{Technisch Ontwerp}
\label{chap:technical}

\section{Software}
\subsection{Hierarchie}

De architectuur van het \autonerf systeem beschreven in sectie
\ref{sec:func-architecture} is opgedeeld in drie logische blokken:

\begin{enumerate}
    \item \texttt{reader};
    \item \texttt{vision};
    \item \texttt{controller};
\end{enumerate}

Deze logische blokken kunnen elk opgedeeld worden in een aantal sub-blokken.
De hierarchie van deze blokken en sub-blokken is te zien in figuur
\ref{fig:hierarchy}.

\begin{figure}[H]
    \begin{center}
        \begin{tikzpicture}[font=\sffamily,
  every matrix/.style={ampersand replacement=\&,column sep=1cm,row sep=1cm},
  source/.style={inner sep=.3cm},
  process/.style={draw,thick,rounded corners,fill=yellow!10,inner sep=.3cm},
  sink/.style={source},
  datastore/.style={draw,very thick,shape=datastore,inner sep=.3cm},
  dots/.style={gray,scale=2},
  to/.style={->,>=stealth',shorten >=1pt,thick,font=\sffamily\footnotesize},
  sub/.style={<-,>=stealth',shorten >=1pt,thick,font=\sffamily\footnotesize},
  every node/.style={align=center,font=\footnotesize}]

    \matrix {
        \node[process](reader){\texttt{reader}}; \& \& \node[process](vision){\texttt{vision}}; \& \& \node[process](control){\texttt{controller}}; \\

        \& \node[process](filter){\texttt{filter}}; \& \& \node[process](oper){Vision operators}; \\
    };

    \draw[to](reader) -- (vision);
    \draw[to](vision) -- (control);

    \draw[sub](reader) |- (filter);
    \draw[sub](vision) |- (oper);
\end{tikzpicture}

    \end{center}
    \caption{De hierarchie van het \autonerf systeem}
    \label{fig:hierarchy}
\end{figure}

\subsection{Architectuur}

De data-flow van het systeem begint bij het \texttt{reader} subsysteem. Dit
subsysteem is verantwoordelijk voor het uitlezen van frames van de camera en
het filteren van deze frames (zie sectie \ref{sub:kleurfilt}) middels het
\texttt{filter} subsubsysteem. Nadat een frame gefiltered is gaat het
gefilterde frame door naar het \texttt{vision} subsysteem. In dit subsysteem
worden een aantal vision operatoren uitgevoerd op het frame om een marker in
het frame te herkennen en de positie hiervan te bepalen, namelijk:

\begin{enumerate}
    \item \texttt{Contrast Stretch}: verbeter het contrast van het frame;
    \item \texttt{Threshold ISO Data}: threshold het frame zodat het een binaire
        afbeelding wordt;
    \item \texttt{Remove Border Blobs}: verwijder blobs die zich aan de rand
        van het frame bevinden;
    \item \texttt{Fill Holes}: vul gaten in blobs op zodat ze minder warrig
        worden;
    \item \texttt{Label Blobs}: geef elke blob een afzonderlijk label zodat ze
        onderschijden kunnen worden;
    \item \texttt{Blob Analyse}: analyseer blobs en berekeen het centrum van
        de grootste blob.
\end{enumerate}

Als het centrum van de grootste blob gevonden is door de \texttt{blob analyse}
operator wordt de grootte van de blob controleerd. Deze moet boven een bepaalde
grens liggen omdat anders ruis ook kan worden gedetecteerd als marker, terwijl
dit natuurlijk niet de bedoeling is. Als er daadwerkelijk een marker is gevonden
wordt de positie van deze doorgegeven aan het \texttt{controller} subsysteem.

Dit subsysteem is verantwoordelijk voor het positioneren en schieten van de
\autonerf launcher. Allereerst wordt de afwijking van de marker tot het centrum
van het frame bepaald in graden. Als deze groter is dan een bepaalde waarde
wordt de positie van de launcher zowel de pan- als tilt-as aangepast. Als de
afwijking lager is dan een bepaalde waarde (bijvoorbeeld 50 pixels) wordt
aangenomen dat de launcher het doel bereikt heeft en dat er geschoten kan
worden. Dit commando wordt gegeven en een uitgang wordt hoog gemaakt waardoor
er een dart afgevuurd wordt.

\subsubsection{\texttt{reader}}

Figuur \ref{fig:ipo-reader} toont het IPO-schema van het \texttt{reader}
subsysteem. De ingang van het systeem is de op het systeem aangesloten camera.
De uitgang van het systeem is een grijswaarde frame wat het van de camera
ontvangen RGB-frame is nadat het is gefiltered.

\begin{figure}[H]
    \begin{center}
        \begin{tikzpicture}[font=\sffamily,
  every matrix/.style={ampersand replacement=\&,column sep=1cm,row sep=2cm},
  source/.style={inner sep=.3cm},
  process/.style={draw,thick,rounded corners,fill=yellow!10,inner sep=.3cm},
  sink/.style={source},
  datastore/.style={draw,very thick,shape=datastore,inner sep=.3cm},
  dots/.style={gray,scale=2},
  to/.style={->,>=stealth',shorten >=1pt,thick,font=\sffamily\footnotesize},
  every node/.style={align=center,font=\footnotesize}]

    \matrix {
        \node[source](in){Frame van USB-camera};
            \& \node[process](proc)[]{Uitlezen frames van USB-camera};
            \&  \node[sink](out)[]{Uitgelezen frame}; \\
    };

    \draw[to](in) -- (proc);
    \draw[to](proc) -- (out);
\end{tikzpicture}

    \end{center}
    \caption{Het IPO-schema van het \texttt{reader} subsysteem}
    \label{fig:ipo-reader}
\end{figure}

\subsubsection{\texttt{vision}}

Figuur \ref{fig:ipo-vision} toont het IPO-schema van het \texttt{vision}
subsysteem. Zoals te zien in het schema heeft het systeem één ingang: het
grijswaarde frame dat gegenereerd is door het \texttt{reader} subsysteem.

Daarnaast heeft het subsysteem twee uitgangen: of er een marker gedetecteerd is
en de coördinaten van de gedetecteerde marker als deze aanwezig is.

\begin{figure}[H]
    \begin{center}
        \begin{tikzpicture}[font=\sffamily,
  every matrix/.style={ampersand replacement=\&,column sep=1cm,row sep=.25cm},
  source/.style={inner sep=.3cm},
  process/.style={draw,thick,rounded corners,fill=yellow!10,inner sep=.3cm},
  sink/.style={source},
  datastore/.style={draw,very thick,shape=datastore,inner sep=.3cm},
  dots/.style={gray,scale=2},
  to/.style={->,>=stealth',shorten >=1pt,thick,font=\sffamily\footnotesize},
  every node/.style={align=center,font=\footnotesize}]

    \matrix {
            \& \& \node[sink](detect){Marker gedetecteerd}; \\
        \node[source](frame){Frame}; \& \node[process](vision){\texttt{vision}}; \\
            \& \& \node[sink](position){Marker positie};\\
    };

    \draw[to](frame)--(vision);
    \draw[to](vision)|-(detect);
    \draw[to](vision)|-(position);
\end{tikzpicture}

    \end{center}
    \caption{Het IPO-schema van het \texttt{vision} subsysteem}
    \label{fig:ipo-vision}
\end{figure}

\subsubsection{\texttt{controller}}

Het IPO-schema van het \texttt{controller} subsysteem (figuur
\ref{fig:ipo-controller}) laat zien dat dit subsysteem één ingang heeft: de
positie van de gedetecteerde marker. Als er in het \texttt{vision} subsysteem
geen marker gedetecteerd is wordt het \texttt{controller} subsysteem niet
aangeroepen.

De uitgangen van het \texttt{controller} subsysteem zijn het eventueel aansturen
van de pan en tilt en het vuren van darts.

\begin{figure}[H]
    \begin{center}
        \begin{tikzpicture}[font=\sffamily,
  every matrix/.style={ampersand replacement=\&,column sep=1cm,row sep=1cm},
  source/.style={inner sep=.3cm},
  process/.style={draw,thick,rounded corners,fill=yellow!10,inner sep=.3cm},
  sink/.style={source},
  datastore/.style={draw,very thick,shape=datastore,inner sep=.3cm},
  dots/.style={gray,scale=2},
  to/.style={->,>=stealth',shorten >=1pt,thick,font=\sffamily\footnotesize},
  every node/.style={align=center,font=\footnotesize}]

    \matrix {
        \node[source](in1){Herkende personenen};\\

        \& \node[process](proc){Bepalen commando's voor Nerfgun};
            \& \node[sink](out){Commando's voor Nerfgun}; \\

        \node[source](in2){Ruimtelijke coordinaten gezichten}; \\
    };

    \draw[to](in1) -| (proc);
    \draw[to](in2) -| (proc);
    \draw[to](proc) -- (out);
\end{tikzpicture}

    \end{center}
    \caption{Het IPO-schema van het \texttt{controller} subsysteem}
    \label{fig:ipo-controller}
\end{figure}

\section{Hardware}
\label{sec:hardware}

\subsection{Opzet}
\label{sub:opzet}
Het initiële ontwerp was een combinatie systeem van FPGA en een CPU. Door in
de FPGA operatoren te implementeren moest er een hoge frame-rate gehaald kunnen
worden. Mede door de communicatie complexiteit is er afgezien van een FPGA
systeem en is er gekozen voor een snelle processor met Linux. Zo kan er gemakkelijk
gebruik worden gemaakt van een USB camera.

\subsection{Ontwikkelingsbord}
\label{sub:devboard}
De markt voor Linux gebaseerde ontwikkelingsbordjes is redelijk uitgebreid. Voor
dit project zijn er echter een aantal eisen.

\begin{itemize}
	\item 2 PWM uitgangen t.b.v. de pan en tilt servo's
	\item minimaal 11 GPIO uitgangen t.b.v. 10 darts + veiligheid
	\item zo hoog mogelijke kloksnelheid t.b.v. frame-processing
	\item genoeg geheugen voor minimaal 2 640*480 RGB frames
	\item een community t.b.v. resources
	\item het moet betaalbaar blijven
\end{itemize}

Gezien het erg belangrijk is om voldoende informatie te kunnen vinden over aansturing
van onder andere hardware en PWM pinnen is er besloten om een BeagleBone Black te
nemen. Met 6 PWM kanalen, 69 GPIO's, 512Mb DDR3 RAM en een 1GHz CPU is dit een prima
apparaat om te gebruiken voor dit project.

\subsection{Cape}
\label{sub:cape}
Om de veiligheid van personen om het systeem te kunnen garanderen wordt er wederom
gebruik gemaakt van een hardware-matig beveiligingssysteem. Bestaand uit een
vereenvoudigde versie van de PC + Arduino combinatie.
De interface zal bestaan uit een 3-tal LEDs en een 2-tal knoppen. De 3 LEDs zullen
de gebruiker laten weten of dat de applicatie aan staat of niet, of dat het apparaat
kan schieten of niet en of dat er herladen moet worden.
Met de 2 knoppen kan het schieten en draaien hardware-matig geactiveerd en gedeactiveerd
worden. Daarnaast zit er een knop op om aan te geven dat het systeem herladen is.
De interface dient als cape op de BeagleBone Black gemonteerd te worden. Verder zal de
cape de nodige voedingsspanningen voorzien aan de kleppen, servo's en de BeagleBone Black
zelf. Door de applicatie automatisch te laten starten op de BeagleBone Black is het doel
een plug-and-play systeem te maken.
