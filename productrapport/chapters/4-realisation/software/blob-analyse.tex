\subsection{Blob analyse}

Na het labelen kunnen de gevonden blobs/objecten worden geanalyseerd. Door
te controleren op massa/grootte van de blob bepaald het systeem waar het op
moet richten. Hierbij wordt de blob met de grootste massa genomen.

Eerst wordt er bij iedere blob bekeken wat de massa is van de blob.

\begin{cppcode}
//First clear the blob_mass registers
for(i = MAX_BLOB_COUNT; i > 0; i--){
    blob_mass[i] = 0;
}

for(i = (FRAME_SIZE - 1); i >= 0; i--){
    if(img[i] > 0){
        blob_mass[img[i]]++;
    }
}
\end{cppcode}

Dan wordt de grootste blob bepaald.

\begin{cppcode}
largest_blob = 0;

for(i = blobCount; i > 0; i--){
    if(blob_mass[i] > largest_blob){
        largest_blob = blob_mass[i];
    }
}
\end{cppcode}

Tot slot moeten de coördinaten van het middelpunt bepaald worden.

\begin{cppcode}
if(imgArr[h][w] == largest_blob){
    //Calculate height
    if(h < min_height){
        min_height = h;
    }
    if(h > max_height){
        max_height = h;
    }
    blob_pos_x = (max_height - min_height) / 2;

    //Calculate width
    if(w < min_width){
        min_width = w;
    }
    if(w > max_width){
        max_width = w;
    }
    blob_pos_y = (max_width - min_width) / 2;
}
\end{cppcode}

Met deze informatie kan de correctie bepaald worden.
