\subsection{Blob analyse}

Na het labelen kunnen de gevonden blobs/objecten worden geanalyseerd. Door
te controleren op massa/grootte van de blob bepaald het systeem waar het op
moet richten. Hierbij wordt de blob met de grootste massa genomen.

Eerst wordt er bij iedere blob bekeken wat de massa is van de blob (listing
\ref{lst:blob-mass}) waarna de grootste blob bepaald kan worden (listing
\ref{lst:blob-size}). Uiteindelijk moet met deze informatie het middelpunt van
de grootste blob bepaald worden (listing \ref{lst:blob-centre}).

\begin{listing}
    \begin{minted}{c}
    for(i = FRAME_SIZE; i > 0; i--){
        blob_mass[img[(i - 1)]] += (img[(i - 1)] != 0)
    }
    \end{minted}
    \caption{Het berekenen van de blob massas}
    \label{lst:blob-mass}
\end{listing}

\begin{listing}
    \begin{minted}{c}
    largest_blob = 0;

    for(i = blobCount; i > 0; i--){
        if(blob_mass[i] > largest_blob){
            largest_blob = blob_mass[i];
        }
    }
    \end{minted}
    \caption{Het vinden van de grootste, aanwezige blob}
    \label{lst:blob-size}
\end{listing}

\begin{listing}
    \begin{minted}{c}
    if(imgArr[h][w] == largest_blob){
        min_height = (h < min_height) ? h : min_height;
        max_height = (h > max_height) ? h : max_height;
        min_width  = (w < min_width)  ? w : min_width;
        max_width  = (w > max_width)  ? w : max_width;

        blob_pos_x = ((max_height - min_height) / 2) + min_height;
        blob_pos_y = ((max_width  - min_width)  / 2) + min_width;
    }
    \end{minted}
    \caption{Bereken van het centrum van een blob}
    \label{lst:blob-centre}
\end{listing}
