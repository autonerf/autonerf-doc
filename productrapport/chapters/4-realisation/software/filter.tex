\subsection{Kleuren filter}
\label{sub:kleurfilt}

Het filteren van de kleuren moet zo snel mogelijk gebeuren omdat het verwerken van
RGB plaatjes erg zwaar is. De kleuren filter zelf is dan ook zo eenvoudig mogelijk
geïmplementeerd. Hieronder is wiskundig te zien hoe er van een RGB plaatje een
grijswaarde wordt gemaakt.

\[ gr(x, y) = g(x, y) - (r(x, y) + b(x, y)) \]

Waarbij $gr$ de grijswaarde van de pixel is; $r$ de rode waarde van de pixel;
$g$ de groene waarde van de pixel; $b$ de blauwe waarde van de pixel; $x$ de x
positie van de pixel en $y$ de y positie van de pixel. Het domein van $gr$ ligt
tussen de 0 en 255. Wanneer een pixel waarde dus onder 0 uit komt wordt deze
gezet op nul.
