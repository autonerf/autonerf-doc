\subsection{Acquisitie}

Het verkrijgen van camera beelden gebeurt middels de V4L2 driver die standaard
met Linux Ångström wordt meegeleverd. V4L staat voor Video4Linux en is een
API die voor een abstractie-laag zorgen voor het communiceren met een groot
aantal verschillende webcams. Omdat veel webcams ieder een andere set commando's
verwachten en hebben is dit erg handig omdat dan veel verschillende webcams
kunnen worden gebruikt.

Voordat er camera beelden kunnen worden gekregen moet eerst de camera
geïnitialiseerd worden (zie bijlage \ref{app:camera-init}). Vervolgens moet de
camera gestart worden waarna frames kunnen worden uitgelezen (listings
\ref{lst:camera-start} en \ref{lst:camera-read}).

\begin{listing}
    \begin{minted}{c}
    int
    camera_start(struct camera_t * camera)
    {
        enum v4l2_buf_type type = V4L2_BUF_TYPE_VIDEO_CAPTURE;

        return ioctl(camera->fd, VIDIOC_STREAMON, &type);
    }
    \end{minted}
    \caption{Het starten van een webcam met Video4Linux}
    \label{lst:camera-start}
\end{listing}

\begin{listing}
    \begin{minted}{c}
    int
    camera_frame_grab(struct camera_t * camera, struct frame_t * frame)
    {
        fd_set              read;
        int                 result;
        struct timeval      timeout;

        do {
            FD_ZERO(&read);
            FD_SET(camera->fd, &read);

            timeout.tv_sec  = 0;
            timeout.tv_usec = 5;

            result = select(camera->fd + 1, &read, NULL, NULL, &timeout);
        } while (result < 0 && (errno = EINTR));

        memset(&frame->buffer, 0, sizeof(struct v4l2_buffer));
        frame->buffer.type   = V4L2_BUF_TYPE_VIDEO_CAPTURE;
        frame->buffer.memory = V4L2_MEMORY_MMAP;

        if (ioctl(camera->fd, VIDIOC_DQBUF, &frame->buffer) < 0) {
            LOG_ERROR(strerror(errno));

            return -1;
        }

        frame->size   = frame->buffer.bytesused;
        frame->pixels = (struct pixel_t *) camera->buffers[frame->buffer.index].data;

        if (camera->filter) {
            frame->filtered = (uint8_t *) calloc(frame->size, sizeof(uint8_t));
            camera->filter(frame, frame->filtered);
        } else {
            frame->filtered = NULL;
        }

        return 0;
    }
    \end{minted}
    \caption{Het uitlezen van een frame van een webcam met Video4Linux}
    \label{lst:camera-read}
\end{listing}

Zoals te zien in de listing in bijlage \ref{app:camera-init} wordt er gebruik
gemaakt van shared memory. Dit betekend dat de V4L driver direct in hetzelfde
geheugen kan schrijven als de \autonerf applicatie wat de leestijd drastisch
omlaag brengt (het geheugen hoeft niet byte voor byte overgezet te worden).
