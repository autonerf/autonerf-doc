\subsection{Remove border blobs}

Op het moment dat er een beeld binnen komt beschouwen we rand objecten als
nutteloos. Simpelweg omdat het niet zeker is wat het voor object is. Daarom
wordt er gebruik gemaakt van de \emph{remove border blobs} operatie.

Er komt een binair plaatje binnen bestaande uit 1-en en 0-en. Door de 1-en
aan de rand van het plaatje te markeren met een ander nummer kan er een
border blob geïdentificeerd worden.

\begin{figure}
    \begin{center}
        \includegraphics[scale=0.5]{figures/border_blob_step1.png}
    \end{center}
    \caption{Rand blobs markeren}
    \label{fig:bbstep1}
\end{figure}

Omdat enkel de rand gescand hoeft te worden kan deze functie redelijk snel
uitgevoerd worden.

\begin{cppcode}
    for(w = width; w >= 0; w--){
        dst->data[0][w]      = dst->data[0][w] * 2;
        dst->data[height][w] = dst->data[height][w] * 2;
    }
\end{cppcode}

Dit kan uiteraard ook even eenvoudig voor de hoogte worden gedaan.

Het voltooien van het markeren is mogelijk door, als er een 1 gezien wordt,
te kijken naar de buren van deze pixel. Als dat een 2 is, dan mag deze pixel
ook gemarkeerd worden als een 2 enzovoort. Dit markeren kan van rechts-onder
naar links-boven en van links-boven naar rechts-onder gedaan worden om het
aantal iteraties te verminderen. Door iedere keer na een markeer ronde te
controleren of er een 1-2 verbinding bij x connected is, kan er bepaald worden
of er nog een keer over het plaatje gegaan moet worden om de overige pixels te
markeren.

\begin{figure}
    \begin{center}
        \includegraphics[scale=0.35]{figures/border_blob_step2.png}
    \end{center}
    \caption{Blijf scannen en markeren tot alles is gemarkeerd}
    \label{fig:bbstep2}
\end{figure}

De scan van links boven naar rechts onder en van rechts onder naar rechts boven
kan in 1 loop uitgevoerd worden.

\begin{cppcode}
    //Lower right -> top left
    if(img[i] == 1){
        if(iNeighbourCount(img, w, h, 2, connected) > 0){
            img[i] = 2;
        }
    }

    //Top left -> lower right
    if(img[size - i] == 1){
        if(iNeighbourCount(img, (width - w), (height - h), 2, connected) > 0){
            img[size - i] = 2;
        }
    }
\end{cppcode}

Tot slot kan er dan met een functie alle pixels met een 2 markeren als een 0.

\begin{figure}
    \begin{center}
        \includegraphics[scale=0.35]{figures/border_blob_step3.png}
    \end{center}
    \caption{Zet alle pixels gemarkeerd met 2 naar 0}
    \label{fig:bbstep3}
\end{figure}


\begin{cppcode}
    if(data[i] == 2){
        data[i] = 0;
    }
\end{cppcode}
