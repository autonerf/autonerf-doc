\subsection{Afwijking berekening}

Het berekenen van de afwijking van een blob ten opzichte van het centrum van
een frame is een redelijk eenvoudige floating point berekening. Voordat de
afwijking berekend kan worden moet de \emph{Field of View} (FOV of $\varphi$)
van de camera bekend zijn. Omdat dit niet in de specificaties van de
gebruikte camera staat is hiervoor een FOV van $45\,^{\circ}$ gebruikt. Hiermee
kan, samen met de resolutie van een frame, het aantal graden per pixels
berekend worden (uitgedrukt in $\phi$):

\[
    \phi = \frac{\varphi}{\sqrt{w^2 + h^2}}
\]

Na het invullen van deze formule komt uit:

\[
    \phi = \frac{45\,^{\circ}}{\sqrt{640^2 + 480^2}} = 0.05625\,^{\circ} \approx 0.06\,^{\circ}
\]

Hieruit blijkt dat er ongeveer $0.06\,^{\circ}$ graden per pixel zijn. Als er
dan een afwijking is van, bijvoorbeeld, 125 pixels op de X-as en -20 pixels op
de Y-as (en we hebben geen margin-of-error ingesteld) dan komt dit overeen met
(ongeveer):

\[
    \phi_x = 0.06\,^{\circ} \times 125 = 7.5\,^{\circ} \\
\]

\[
    \phi_y = 0.06\,^{\circ} \times -20 = -1.2\,^{\circ}
\]

Het is ook mogelijk om een \emph{margin-of-error} in te stellen, wat betekend
dat er een ruimte om het centrum van het frame heen gezet wordt waarin, als de
afwijking kleiner is dan dat, het als ``goed genoeg'' beschouwd kan worden.

Door de bovenstaande formules te gebruiken in combinatie met een margin of
error kan bijvoorbeeld gezegd worden dat de $\phi_y$ goed is en alleen de X-as
(pan of $\phi_x$) bijgesteld moet worden.
