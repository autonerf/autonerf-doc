\subsection{Contrast stretch}

Om een betrouwbaardere threshold te kunnen maken gebruiken we een contrast stretch
operatie. Deze functie bepaald de laagste en hoogste waarde in het beeld en smeert
deze uit over het complete grijswaarde gebied.

Door een vermenigvuldigingsfactor te bepalen is het eenvoudig om dit te
bereiken. De vermenigvuldigingsfactor zal uiteraard groter dan 1 moeten zijn,
maar kan ook een komma getal zijn. Float operaties zijn dus essentieel en
daarbij is een FPU (Floating Point Unit) erg handig. Om de zware calculaties te
verminderen wordt er gebruik gemaakt van een LUT (Look Up Table) die hetzelfde
bereik heeft als de grijswaarde. Gezien er gebruik wordt gemaakt van een
grijswaarde bereik van 256 moet de calculatie 256x worden uitgevoerd.

De verhouding wordt als volgt bepaald:

\[ R = \frac{D}{max - min} \]

Hierin is $R$ de vermenigvuldigingsfactor is; $D$ het grijswaarde bereik; $max$
is de maximale pixel waarde in het originele beeld en $min$ de minimale pixel
waarde in het originele beeld.

Daarna kan de LUT gevuld worden, geïllustreerd door de volgende pseudo code:

\begin{minted}{text}
    value = i * R;
    if value > D then value = D
    lut[i] = value;
\end{minted}

Tot slot kan er eenvoudig vanuit de huidige waarde in de LUT de nieuwe pixel
waarde worden opgevraagd. Dit wordt over het hele beeld gedaan om een verbeterd
contrast te krijgen.
