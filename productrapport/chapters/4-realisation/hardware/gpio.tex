\subsection{GPIO controller}
\label{sec:gpio}

Om GPIO's aan te sturen op de Beagle Bone Black dient de modus ingesteld te worden
samen met PULL-UP/PULL-DOWN weerstanden. Daarnaast dienen deze na iedere re-boot
weer ge-exporteert te worden om vervolgens te bepalen of het een input/output moet
zijn en om de waarde van de GPIO te wijzigen.

Het instellen van de pin modus en PULL-UP en PULL-DOWN weerstanden dient gedaan te
worden in de overlay. Dit is een .dts file waarin onder andere de volgende informatie 
staat.

\begin{minted}{text}
/* <Pin offset adres> <Value> </* pin comment */>*/
          0x0c0        0x07    /* 78: OUTPUT MODE7 + PULL-DOWN = Missile 1     */
\end{minted}

De waarden en pin offsets kunnen uit de tabellen in bijlage \ref{app:gpio} gehaald worden. Hier
moet dus ook een PWM pin ingesteld worden indien deze nodig is!

Vervolgens dient de .dts file gecompileerd te worden naar .dtbo. Hieronder is het
commando te vinden er vanuit gaande dat de .dts file DM-GPIO-Test.dts heet.

\begin{minted}{sh}
dtc -O dtb -o DM-GPIO-Test-00A0.dtbo -b 0 -@ DM-GPIO-Test.dts
\end{minted}

Deze .dtbo file dient vervolgens gekopieerd te worden naar de /lib/fimware folder.
Om deze te activeren is er een re-boot nodig en vervolgens kan deze aan de device 
tree worden toegevoegd middels het commando hieronder.

\begin{minted}{sh}
echo DM-GPIO-Test > /sys/devices/bone_capemgr.8/slots
\end{minted}

Uiteraard kan ook dit automatisch bij het opstarten gebeuren door deze in de
\texttt{uEnv.txt} file te zetten. In sectie \ref{sec:pan-tilt-control} staat
beschreven hoe.

Omdat er pinnen worden gebruikt die ook voor de HDMI interface worden gebruikt 
is het nodig om de HDMI interface uit te zetten in de device tree.

Om een GPIO te kunnen gebruiken dienen de volgende commando's uitgevoerd te worden.
Deze commmando's worden dan ook afgehandeld door de gpio.c file.

\begin{minted}{sh}
echo 60 > /sys/class/gpio/export
echo out > /sys/class/gpio/gpio60/direction
echo 1 > /sys/class/gpio/gpio60/value
\begin{minted}

Dit maakt de uitgang op GPIO 60 hoog. Laag is natuurlijk eenvoudig te realiseren
door 0 naar value te schrijven.
Het gebruik maken van een input is eenvoudig door \emph{in} naar direction te schrijven.
Dan is de waarde uit te lezen in value.

Als het programma afsluit is het netjes om de GPIO's ook weer te \emph{unexporten}.

\begin{minted}{sh}
echo 60 > /sys/class/gpio/unexport
\end{minted}

% Bron: http://derekmolloy.ie/beaglebone/beaglebone-gpio-programming-on-arm-embedded-linux/
