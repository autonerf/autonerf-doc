\subsection{Pan en tilt controller}
\label{sec:pan-tilt-control}

De pan en tilt controller zorgt er voor dat de servo.c file met de juiste outputs
wordt bestuurd met de juiste waarden. De servo.c file stuurt op zijn beurt de pwm.c
file aan met de juiste frequentie en dutycycles (zet de hoek om in de juiste
dutycycle). De \texttt{pwm.c} file zorgt er voor dat alle data naar de juiste
file wordt gestuurd, outputs niet dubbel geconfigureerd zijn, etcetera.

Om een PWM signaal te genereren dient de PWM module geactiveerd te worden. Hoe
dit kan in Ångström staat beschreven in listing \ref{lst:pwm-activation}. Het
voorbeeld is voor pin texttt{P8.13}.

\begin{listing}
    \begin{minted}{sh}
    echo am33xx_pwm > /sys/devices/bone_capemgr.8/slots
    echo bone_pwm_P8_13 > /sys/devices/bone_capemgr.8/slots
    echo 20000000 > /sys/devices/ocp.2/pwm_test_P8_13.15/period
    echo 10000000 > /sys/devices/ocp.2/pwm_test_P8_13.15/duty
    \end{minted}
    \caption{Activeren en instellen PWM}
    \label{lst:pwm-activation}
\end{listing}

% Bron: http://www.phys-x.org/rbots/index.php?option=com_content&view=article&id=106:lesson-3-beaglebone-black-pwm&catid=46:beaglebone-black&Itemid=81

Belangrijk is dat de pinmode op PWM ingesteld staat. Dit kan worden gedaan
middels een overlay. Meer hier over in GPIO controller (sectie \ref{sec:gpio}).

De \texttt{am33xx\_pwm} en \texttt{bone\_pwm\_P8\_13} kunnen standaard in slots
worden gezet. Dit kan bereikt worden door in \texttt{uEnv.txt} het volgende te zetten:

\begin{minted}{text}
    capemgr.enable_partno=am33xx_pwm,bone_pwm_P8_13
\end{minted}

\texttt{uEnv.txt} bevindt zich op de virtuele SD kaart die automatisch verbind
met de computer (via USB) tijdens het booten.

De periode en dutycycle worden ingesteld via de \texttt{pwm.c} file. Daarnaast
bouwt deze op basis van het GPIO nummer automatisch de juiste directory op. Het
enige wat soms verandert is het suffix nummer van de directory (bijvoorbeeld
\texttt{P8\_13.xx/}). Dit kan echter aangepast worden in de \texttt{pwm.c} file.
