\chapter{Conclusie en Aanbevelingen}
\label{chap:conclusion}

% Wat is wel en wat is niet gerealiseerd?
% Wat kan er aan het product worden aangevuld, uitgebreid, verbeterd?

Doordat er in een later stadium is besloten geen FPGA te gebruiken voor 
acquisitie en filtering hebben we een ander product opgeleverd als in 
het plan van aanpak beschreven staat. Achteraf gezien is dat de juiste 
beslissing geweest gezien er te veel lastige onderdelen in dat idee 
zaten.
De aanpak met de BeagleBone Black is goed gelukt gezien er veel informatie 
beschikbaar was rondom het onderwerp. Code voor externe communicatie 
(frame grabber camera, I/O's en PWM aansturen) was prima te realiseren 
ondanks dat de frame grabber af en toe wat instabiel was.
Dankzij de goede ondersteuning van diverse vakken is het goed gelukt de 
informatie uit het beeld te filteren.

Op het moment van schrijven zijn er wel uitbreidingen op het systeem aan 
te bregen. Zo kan het object nog gedetecteerd worden middels een detectie 
methode, bijvoorbeeld formfactor of de moment methode. Daarnaast kan er 
gebruik worden gemaakt van andere (meerdere) of instelbare kleuren filters. 
Ook is de snelheid van operatoren nog op te voeren door gebruik te maken 
van de NEON instructies (SIMD - Singele Instruction Multiple Data). Zo 
kan de response van het systeem verbeterd worden.