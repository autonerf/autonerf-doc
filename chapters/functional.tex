\chapter{Functioneel Ontwerp}
\label{ch:functional}

Dit hoofdstuk definiëerd de functionele specificatie van het Autonerf project.
Het beschrijft een aantaal functionele eisen en de functionele architectuur
van het systeem.

\section{Specificatie}

Het te ontwikkelen systeem moet aan de volgende eisen voldoen:

\begin{itemize}
    \item Het systeem moet kunnen werken op een gewone computer;
    \item Het systeem moet personen herkennen en daar \emph{darts} op af vuren;
    \item Het systeem moet (om personen te herkennen) gebruik maken van
        real-time camera beelden;
    \item Het systeem moet in staat zijn op bewegende doelwitten te kunnen vuren.
\end{itemize}

\section{Architectuur}
\label{sec:func:arch}

De omschrijving van de architectuur van het Autonerf systeem begint bij het
globale input-proces-output (IPO) schema (te zien in figuur \ref{fig:global-ipo}).
Dit schema toont de inputs van het systeem (de real-time camera beelden) en de
output van het systeem (de dart die mogelijk afgevuurd wordt).

\begin{figure}[H]
    \begin{center}
        \input{figures/ipo/global.tex}
    \end{center}
    \caption{Het globale input-process-output schema van het Autonerf systeem}
    \label{fig:global-ipo}
\end{figure}

\vfill
\pagebreak

Om de taak goed te kunnen uitvoeren heeft het systeem de volgende logische
blokken nodig:

\begin{enumerate}
    \item De \emph{reader}: verantwoordelijk is voor de acquisitie van
        frames vanuit de real-time feed van de camera beelden;
    \item De \emph{detector}: verantwoordelijk voor het detecteren van gezichten
        in een frame;
    \item De \emph{recognizer}: verantwoordelijk voor het herkennen van gezichten
        en deze aan bepaalde personenen koppelen zodat hiermee een conclusie
        uit kan worden getrokken;
    \item De \emph{localizer}: verantwoordelijk voor het bepalen van de locatie
        en radiale afwijking van gezichten op basis van ontvangen frames;
    \item De \emph{controller}: verantwoordelijk voor het aansturen van de
        Nerfgun.
\end{enumerate}

\begin{figure}[H]
    \begin{center}
        \begin{tikzpicture}[font=\sffamily,
  every matrix/.style={ampersand replacement=\&,column sep=1cm,row sep=1cm},
  source/.style={inner sep=.3cm},
  process/.style={draw,thick,rounded corners,fill=yellow!10,inner sep=.3cm},
  sink/.style={source},
  datastore/.style={draw,very thick,shape=datastore,inner sep=.3cm},
  dots/.style={gray,scale=2},
  to/.style={->,>=stealth',shorten >=1pt,thick,font=\sffamily\footnotesize},
  every node/.style={align=center,font=\footnotesize}]

    \matrix {
        \node[process](read){Reader};
            \& \node[process](detect){Detector};
            \& \node[process](local){Localizer};
            \& \node[process](control){Controller}; \\
    };

    \draw[to](read) -- (detect);
    \draw[to](detect) -- (local);
    \draw[to](local) -- (control);
\end{tikzpicture}

    \end{center}
    \caption{De logische blokken van het Autonerf systeem}
    \label{fig:func-architecture}
\end{figure}
