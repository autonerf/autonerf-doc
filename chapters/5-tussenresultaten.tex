\chapter{Tussenresultaten}

Er zullen gedurende het project verschillende tussendocumenten worden
opgeleverd, namelijk:

\begin{itemize}
    \item Het functioneel ontwerp;
    \item het vision ontwerp;
    \item het SDE-document;
    \item en het vision rapport.
\end{itemize}

\section{Functioneel ontwerp}

Het functioneel ontwerp zal beschrijven wat het systeem moet doen. Er wordt in
dit document niet ingegaan op hoe het systeem zijn taken zal uitvoeren. Er zal
in dit document een globaal input-process-output (IPO-)schema worden beschreven
waarna de functionele specificatie tot in details uitgewerkt wordt.

\section{Vision ontwerp}


Het vision ontwerp beschrijft het ontwerp van de vision operatoren aan de hand
van de volgende vijf stappen:

\begin{enumerate}
    \item Acquisition
    \item Enhancement
    \item Segmentation
    \item Feature extraction
    \item Classification
\end{enumerate}

\section{SDE-document}

In dit (Software Development Environment) document zal beschreven worden 
\emph{welke} software omgeving gebruikt wordt voor de ontwikkeling van de
software.

\section{Vision rapport}

“De vision realisatie beschrijft de manier waarop het resultaat tot stand is
gekomen. Welke operatoren zijn er toegepast, waarom is hiervoor gekozen, wat
zijn de alternatieven, etc. Hierin wordt per stap op een heldere manier met
tussenstappen en voorbeeldplaatjes uitgelegd hoe een visionstap geïmplementeerd
is. Hieronder staat e.e.a. schematisch weergegeven, waarin iedere pijl een
vision bewerking voorstelt. Daarnaast kan er in het document achtergrond
informatie worden gegeven over de bewerkingen met daarbij de juiste referentie
naar de bron.”\footnote{Bron: EVD 2013-2014 Projectdocumentatie (p. 2),
geraadpleegd op: \today}
